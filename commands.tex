\newcommand{\characterLine}[3]{%
    \begin{tblr}{ X[3,r] X[1,c] X[9,l] }
        \textbf{#1} & \underline{\textit{#2}} & #3 \\ % Render character name and type
    \end{tblr}
}

\newcommand{\newpart}{%
    \noindent\makebox[\linewidth]{\rule{\maxdimen}{0.2pt}} \\[5mm]
}

\newcommand{\nameformat}[3]{%
    \textcolor[HTML]{#3}{\colorbox[HTML]{#2}{#1}}
}


\begin{pycode}
def get_character(a):
    colorId = 0
    for alias, character in characters.items():
        if alias == a:
            name = character.get("name", "Unknown")
            return f'\\nameformat{{{name}}}{{{colors[colorId][1]}}}{{{colors[colorId][0]}}}'
            break

        colorId += 1
    else:
        return f'\\nameformat{{UNKNOWN}}{{{colors[0][1]}}}{{{colors[0][0]}}}'
\end{pycode}

\newcommand{\postava}[1]{
    \pyc{%
    print(get_character("#1"))
    }
}